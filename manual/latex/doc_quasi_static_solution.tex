
\section{Quasi-static solution with local damping}\label{sec:quasi-static-solution-with-local-damping}

The local damping is used to get a quasi-static solution of the dynamic system using a viscous nodal force. In each time step, a viscous force is applied in each node, whose magnitude is proportional and opposite to the nodal velocity.

$$ f_{iI}^{dnplocal} = - \alpha |f_{iI}^{unb}| \hat{v}_{iI} $$

, where the unbalanced nodal force is:

$$ f_{iI}^{unb} =  f_{iI}^{int} + f_{iI}^{ext} $$

Therefore, the resulting discrete form of the motion equation with viscous nodal damping is:

$$ \dot{p}_{iI} = f_{iI}^{int} + f_{iI}^{ext} + f_{iI}^{dnplocal} $$
