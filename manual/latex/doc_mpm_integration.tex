\section{Explicit MPM integration}\label{sec:explicit-mpm-integration}

The discrete form of the motion equation needs to be integrated in time for obtaining the solution in time $t^{n+1}$. The displacement, the velocity and the acceleration at time $t^0, t^1, t^2, ... , t^n$ are knows, and the values at time $t^{n+1}$ are required, namely the solution of the problem. 
The time integration can be done by explicit and implicit methods. In explicit method, the solution $t^{n+1}$ is obtained only with the current information f($t^{n},..., t^{0}$). In implicit method the solution needs to solve a system due the solution is in function of the form f($t^{n+1},...,t^{0}$).

\subsection{Central difference Method}

In central difference method, the velocity at $t^{n+1/2}$ can be approximated as:

$$ \dot{u}^{n+1/2} = (u^{n+1} - u^{n}) \Delta t $$

and, the acceleration in $t^{n}$ can be approximated as:

$$ \ddot{u}^{n} =  (\dot{u}^{n+1/2} - \dot u ^ {n-1/2})\Delta t $$

and therefore, the required displacement at $t^{n+1}$ can be calculated as: $ u^{n+1} = u^{n} + \dot u ^ {n+1/2} \Delta t $, where the velocity is

$$ \dot u^{n+1/2} = \dot{ u } ^ {n-1/2} + \ddot u ^ {n}  \Delta t $$.

The motion equation in time $ t^{n} $ is $ m \, \ddot u ^{n} = f ^{n} $, therefore the acceleration in time $t^{n}$ is $ \ddot u ^{n} = f ^{n} / m $. Using this acceleration equation in the $\dot u^{n+1/2}$ we have the velocity $\dot u ^ {n+1/2}$:

$$
\dot u ^ {n+1/2} = \dot u ^ {n-1/2} + f ^{n} / m \, \Delta t
$$

\section{Numerical implementation of central difference method}\label{sec:numerical-implementation-of-central-difference-method}

For one $\Delta t$, the updated position can be obtained as:

\begin{algorithm}
	\caption{Explicit time integration}
	\begin{algorithmic}[1]
		\State Compute forces $f^n$
		\State Compute acceleration $\ddot{u}^n = f^n/m$
		\State Update velocity $\dot{u}^{n+1/2} = \dot{u}^{n-1/2} + \ddot{u}^n \Delta t$
		\State Update position $u^{n+1} = u^n + \dot{u}^{n+1/2}\Delta t$
		\State $n \leftarrow n + 1$
	\end{algorithmic}
\end{algorithm}

\section{Stability Requirement}\label{sec:stability-requirement}

The central difference method is explicit here and conditionally stable, so the time step must be less that a certain value for avoiding error amplification. For linear systems this critical time step value depends on the natural period of the system. For undamped linear systems the critical time step is: $ \Delta t_{cr} = T_n / \pi$, where $ T_n  $ is the smallest natural period of the system. For finite element method, the critical time step of the central difference method can be expressed as:

$$ \Delta t_{cr} = \text{min}_e (l^e/c)$$

Where $l^e $ is the characteristic length of the element and $c$ is the sound speed. This time step restriction implies that time step has to be limited such that a disturbance, a mechanical wave, can travel across the smallest characteristic element length withing a single time step. 

This condition is known as CFL condition, or Courant-Friedrichs-Lewy condition. For linear elastic material the sound speed (compression P wave) is:

$$ c = \sqrt{\frac {E (1-\nu)} {(1+\nu)(1-2\nu) \rho} }$$

In the MPM, the particles can has velocities in any time step, so the critical time speed can be written with this velocity plus:

$$ \Delta t_{cr} = l^e / \text{max}_p ( c_p + |v_p| )$$

In a structured regular mesh, $l^e$ is the grid cell dimension. And $c_p$ is the sound speed calculated with the material parameters stored in particles.
