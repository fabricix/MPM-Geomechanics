\section{Execution}

In order to run simulations in several terminal, you can add the compiled code in the system `PATH`.

For Windows, you can follow these steps:
\begin{itemize}
	\item Open the Start Search, type in "env", and select "Edit the system environment variables"
	\item In the System Properties window, click on the "Environment Variables..." button.
	\item In the Environment Variables window, under the "System variables" section, find and select the "Path" variable, then click on the "Edit..." button.
	\item In the Edit Environment Variable window, click on the "New" button and add the path to the folder where the compiled binary is located (e.g., \verb|C:\path\to\MPM-Geomechanics\build\CMake\build\Release|).
	\item Click "OK" to close all windows.
\end{itemize}

For Linux, you can add the following line to your \verb|~/.bashrc| or \verb|~/.bash_profile| file:

\begin{lstlisting}
	export PATH=$PATH:/path/to/MPM-Geomechanics/build/CMake/build
\end{lstlisting}

For both Windows and Linux, after adding the path, you can open a new terminal and run the program from any location by simply typing `MPM-Geomechanics` followed by the input file name.