\subsection{Base Acceleration Example}
\label{sec:base_acceleration_example}

\subsubsection*{Introduction}

In this example we model the motion of an elastic body subjected to a dynamic boundary condition.  
The body is a cuboid with dimensions \( l_x = l_y = 0.3\,\text{m} \) and \( l_z = 0.8\,\text{m} \),  
with lower coordinate point \( p_{min} = (0.4, 0.4, 0.0)\,\text{m} \).

The base acceleration is defined as:
\[
\ddot{u} = A\,2\pi f \cos(2\pi f t + \alpha)
\]

The total simulation time is \( T = 2\,\text{s} \), with a time step \( \Delta t = 10^{-4}\,\text{s} \).  
Material properties:
\[
\rho = 2500\,\text{kg/m}^3, \quad E = 100 \times 10^6\,\text{Pa}, \quad \nu = 0.25
\]

\begin{figure}[t!]
	\centering
	\includegraphics[width=0.8\textwidth]{figures/geometry-body.png}
	\caption{Geometry of the elastic body.}
	\label{fig:geometry_body}
\end{figure}

\subsubsection*{MPM Model}

The MPM model consists of uniformly distributed particles inside the body, created with the keywords  
\texttt{"body"} and \texttt{"cuboid"}. The mesh dimensions are \( \Delta x = \Delta y = \Delta z = 0.1\,\text{m} \),  
and it covers the full expected displacement range of the body.  
The mesh uses \( n_x = 12, n_y = 12, n_z = 15 \).

\subsubsection*{Input File}

\begin{lstlisting}
	{
		"stress_scheme_update":"USL",
		"shape_function":"GIMP",
		"time":10,
		"time_step":0.0005,
		"gravity":[0.0,0.0,0.0],
		"n_threads":1,
		"damping": {
			"type":"local",
			"value":0.0
		},
		"results": {
			"print":100,
			"fields":["id","displacement","velocity","material","active","body"]
		},
		"n_phases":1,
		"mesh": { 
			"cells_dimension":[0.1,0.1,0.1],
			"cells_number":[10,10,15],
			"origin":[0,0,0]
		},
		"earthquake": {
			"active": true,
			"file": "base_acceleration.csv",
			"header": true
		},
		"material": {
			"elastic_1": {
				"type":"elastic",
				"id":1,
				"young":10e6,
				"density":2500,
				"poisson":0.25
			}
		},
		"body": {
			"columns_1": {
				"type":"cuboid",
				"id":1,
				"point_p1":[0.2,0.2,0],
				"point_p2":[0.5,0.5,1.0],
				"material_id":1
			}
		}
	}
\end{lstlisting}

\subsubsection*{Earthquake Block Parameters}

\begin{lstlisting}
	"earthquake": {
		"active": true,
		"file": "base_acceleration.csv",
		"header": true
	}
\end{lstlisting}

Where:
\begin{itemize}
	\item \textbf{active}: Enables or disables seismic loading.
	\item \textbf{file}: Path to the CSV file containing time, acceleration\_x, acceleration\_y, and acceleration\_z.
	\item \textbf{header}: True if the CSV has a header row.
\end{itemize}

Example of the first lines of the record:

\begin{lstlisting}
	t,ax,ay,az
	0.0,-1.8849555921538759,-0.9424777960769379,-0.0
	5e-05,-1.8849554991350466,-0.9424777844495842,-0.0
	0.0001,-1.884955220078568,-0.9424777495675233,-0.0
	0.00015000000000000001,-1.8849547549844674,-0.9424776914307561,-0.0
	...
\end{lstlisting}

\subsubsection*{Post-processing (results and visualizations) }

Particle results are found in the \texttt{/particle} folder and grid results in the \texttt{/grid} folder located inside the current working directory. Particle results(\texttt{particle\_1.vtu}, \texttt{particle\_2.vtu}, ..., \texttt{particle\_41.vtu})  
are referenced in \texttt{particleTimeSerie.pvd}, which can be opened in ParaView: \verb|File - Open - particleTimeSerie.pvd|The mesh can be loaded with  \verb|File - Open - eulerianGrid.vtu |

\begin{figure}[h!]
	\centering
	\includegraphics[width=0.75\textwidth]{figures/mpm-model-particles-and-mesh.png}
	\caption{Particles and mesh of the analyzed case.}
	\label{fig:mpm_particles_mesh}
\end{figure}

\subsubsection*{Verification of Dynamic Boundary Condition}

The velocity from the analytical input function
\[
\dot{u} = A \sin(2 \pi f t + \alpha)
\]
was compared with the particle velocity at the base of the model.  
The results show excellent agreement between analytical and MPM-calculated velocities.

\begin{figure}[H]
	\centering
	\includegraphics[width=0.75\textwidth]{figures//velocity-base-verification.png}
	\caption{Verification of velocities obtained with MPM simulation.}
	\label{fig:velocity_base_verification}
\end{figure}
