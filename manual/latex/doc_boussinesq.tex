\subsection{Boussinesq's Problem}
\label{sec:boussinesq_problem}

\subsubsection*{Introduction}

In Geomechanics, the Boussinesq's problem refers to the point load acting on a surface of an elastic half-space.  
The boundary conditions for this problem are:

\begin{itemize}
	\item The load \( P \) is applied only at one point, at the origin.
	\item The load is zero at any other point.
	\item For any point infinitely distant from the origin, the displacements must all vanish.
\end{itemize}

\begin{figure}[h!]
	\centering
	\includegraphics[width=0.45\textwidth]{figures//boussinesq-problem.png}
	\caption{Boussinesq's problem.}
	\label{fig:boussinesq_problem}
\end{figure}

\subsubsection*{Analytical Solution}

The analytical solution of the vertical displacement field is:
\[
u_z(x,y,z) = \frac{P}{4 \pi G d} \left( 2 (1-\nu) + \frac{z^2}{d^2} \right)
\]
where \( G = \frac{E}{2(1+\nu)} \) is the shear modulus of the elastic material,  
\( \nu \) is the Poisson ratio, and \( d = \sqrt{ x^2 + y^2 + z^2 } \) is the total distance from the load to the point.

\subsubsection*{MPM Model and Result Comparison}

To model the displacement field generated by the point load, we create an elastic body with dimensions  
\( l_x = l_y = l_z = 1\,\text{m} \), using the keyword \texttt{"cuboid"} with Point 1 at (0,0,0) and Point 2 at (1,1,1).

For the elastic parameters:
\[
E = 200 \times 10^6\,\text{Pa}, \quad \rho = 1500\,\text{kg/m}^3, \quad \nu = 0.25
\]

The computational mesh has cell dimensions \( \Delta x = \Delta y = \Delta z = 0.1\,\text{m} \).  
A nodal load of magnitude 1 is applied in the vertical direction at the midpoint of the upper surface.  
The plane \( Z_n \) is free, while all other planes are sliding (only tangential displacements allowed).  
Dynamic relaxation is used to reach a static solution, through the keyword \texttt{"damping"} with type \texttt{"kinetic"}.

\subsubsection*{Input File}

\begin{lstlisting}
	{
		"body": {
			"elastic-cuboid-body": {
				"type":"cuboid",
				"id":1,
				"point_p1":[0.0,0.0,0],
				"point_p2":[1,1,1],
				"material_id":1
			}
		},
		"materials": {
			"material-1": {
				"type":"elastic",
				"id":1,
				"young":200e6,
				"density":1500,
				"poisson":0.25
			}
		},
		"mesh": { 
			"cells_dimension":[0.1,0.1,0.1],
			"cells_number":[10,10,10],
			"origin":[0.0,0.0,0.0],
			"boundary_conditions": {"plane_Zn":"free"}
		},
		"time":0.025,
		"time_step_multiplier":0.3,
		"nodal_point_load": [[[0.5, 0.5, 1.0], [0.0, 0.0, -1.0]]],
		"damping": {"type":"kinetic" }
	}
\end{lstlisting}

The MPM numerical results show good agreement with the analytical solution,  
with small deviations due to discretization and the representation of the domain by particles with finite volume.  
Errors decrease with mesh refinement.

\begin{figure}[t!]
	\centering
	\includegraphics[width=0.75\textwidth]{figures/boussinesq-problem-verification.png}
	\caption{Comparison between analytical and MPM results.}
	\label{fig:boussinesq_verification}
\end{figure}